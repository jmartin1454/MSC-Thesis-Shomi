%your front matter - fill in your personal details here!
%%

\title{Active Magnetic Compensation Prototype for Neutron Electric Dipole Moment Experiment}
\author{Shomi Ahmed}

\degreemonth{March} % month final submission occurs.
\degreeyear{2019}%
\degree{Master of Science}%
\department{Physics & Astronomy}%
\advisor{Dr. Jeff Martin and Dr. Chris Bidinosti} %



%\copyrightpage
% Insert a blank page for two-sided
%  \newpage
%  \thispagestyle{empty}
%  \hbox{}
%  \newpage


\maketitle

\begin{abstract}
The existence of a non-zero neutron electric dipole moment (nEDM) would violate parity and time-reversal symmetry.  Extensions to the Standard Model predict the nEDM to be $10^{-26}$ -- $10^{-28}$~e$\cdot$cm.  The current best upper limit set by Sussex/RAL/ILL nEDM experiment is $3.0 \times 10^{-26}$~e$\cdot$cm\cite{bestLim_1,bestLim_2}.  The nEDM experiment at TRIUMF is aiming at the $10^{-27}$~e$\cdot$cm sensitivity level.  We are developing the world's highest density source of UCN.  The experiment requires a very stable ($<$~pT) and homogeneous ($<$~nT/m) magnetic field ($B_0$) within the measurement cell.  

My involvement in the nEDM experiment is the development of active magnetic shielding to stabilize the external magnetic field by compensation coils. I have optimized a prototype active magnetic shield at The University of Winnipeg. I have also simulated the behaviour of the coils in the presence of the passive magnetic shields  using finite element analysis (FEA), and made comparisons with  experimental results  to test the successfulness of the control system. A major challenge of the active 
compensation system is its  slow current response.  This is now understood and several recommendations are made to improve the performance in future realizations of such a system.

%  This thesis discussed my journey to understand the system and recommendations for future researchers in active compensation field.



% .Moreover, the magnetic environment at TRIUMF is more challenging than in our lab in Winnipeg, because of the closeness of the experiment to the TRIUMF cyclotron (B $\sim 350 - 400$ $\mu$T 'which is almost one order of magnitude larger than usual background fields') and the changing environment with iron.  Studies of the implementation at TRIUMF will also be reported.

\end{abstract}

\newpage
\tableofcontents
\addcontentsline{toc}{section}{Table of Contents}
%comment these out if you don't want a detailed list of figures and tables!
\listoffigures
\listoftables

\begin{acknowledgments}
\vspace{2em}

I am an engineering student about to finish my Masters in Physics and most of you are familiar with the opposite trend!! But this is happening only because of the mercy of the almighty Allah with my advisor Dr. Jeff Martin as the medium. I would be ever grateful to Dr. Martin for taking the risk, for believing in me, giving me the chance and always be kind with me throughout my Masters journey. He gave me the unparalleled freedom to make a foreign land which is 12000~km away from my motherland to call as my second home. I must say all of my friends are jealous of my super kind advisor. He was the one who always gave me ideas whenever I was stucked in the research and helped me solve the problems. He gave me the opportunity to discover all over Canada attending conferences beyond a Masters student can imagine. He was super helpful and patience throughout my thesis writing journey. I have learned a lot from him.

Next important person who made my Masters life smoother is David Ostapchuk. He is a very talented person with enormous patience. I must confess I had absolutely no idea how and from where to begin my research and he is the person who gave most of the hours of a day for continuous three months to help me settle in and understand the project. In addition to my advisor whenever I faced problems, I bothered Dave and he helped me graciously.

I would also like to thank my co-advisor Dr. Chris Bidinosti, for his generous help and support during my graduate studies. I am grateful to Dr. Russell Mammei to help me understand OPERA simulation by working with me and providing me his own codes. I am also grateful to Dr. Beatrice Franke and Dr. Blair Jamieson for their valuable suggestions during my research work. 

The graduate students specially Dr. Taraneh Andalib, Sakib Rahman, Michael Loren Lang, Moushumi Das, Ray Dwaipayan, Sean Hansen-Romu and Wolfgang Klassen were super helpful to me and I am thankful to them. 

Last but the least, I want to thank my family and friends who have supported me always. I want to dedicate the Masters work to my father whom I lost during my Masters journey. May Allah rest his soul and place him in heaven.

\vspace{2em}
Thank You All.

-Shomi

\end{acknowledgments}

%
%\quotation
%\begin{quote}
%\hsp \em Is it the God's will or the Lotus flower's intention,  when it blooms in the mud?
%\end{quote}



%\dedication
%\vspace*{\fill}
%%\begin{center}
%\begin{quote}
%\hfil \hsp \Large \em Dedicated to all the sacrifices ...\hfil
%\end{quote}
%\vspace*{\fill}
%\end{center}
%\newpage
%\cleardoublepage
%\thispagestyle{empty}
\startarabicpagination
%Use 'startarabicpagination' to use both numbers and roman letters for pages
%%% end
