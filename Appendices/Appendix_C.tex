\section{Uniform internal field generation}
From ~\cite{ smythe}, the field in the region $r< a$ for a current loop at \(\theta=\alpha\) is,
\begin{equation}\label{bbr}
B_r=\frac{\muo I \sin \alpha}{2a}\sum_{n=1}^{\infty} \left(\frac{r}{a}\right)^{l-1}P_l^1(\cos\alpha) P_l(\cos\theta)
\end{equation}
\begin{equation}\label{bbt}
B_\theta=-\frac{\muo I \sin \alpha}{2a}\sum_{n=1}^{\infty} \frac{1}{n}\left(\frac{r}{a}\right)^{l-1}P_l^1(\cos\alpha) P_l^1(\cos\theta)
\end{equation}

For Helmholtz coils of radius \(r_c\) located at \(z=\pm z_o\) i.e. for two loops at \(\theta=\alpha, \pi-\alpha \), \(I_2=I_1=I\) and \(z_o=\frac{r_c}{2}\),
\begin{equation}\label{sinH}
\sin\alpha=\frac{r_c}{a}=\frac{r_c}{\sqrt{z_o^2+r_c^2}}=\frac{r_c}{\sqrt{\left(\frac{r_c}{2}\right)^2+r_c^2}}=\frac{r_c}{\sqrt{\frac{5r_c^2}{4}}}=\frac{2}{\sqrt{5}}
\end{equation}
\begin{equation}\label{sinHb}
\sin(\pi-\alpha)=\sin\pi\cos\alpha-\cos\pi\sin\alpha=\sin\alpha=\frac{2}{\sqrt{5}}
\end{equation}

\begin{equation}\label{cosH}
\cos\alpha=\frac{z_o}{a}=\frac{r_c}{2\sqrt{z_o^2+r_c^2}}=\frac{r_c}{2\sqrt{\left(\frac{r_c}{2}\right)^2+r_c^2}}=\frac{r_c}{2\sqrt{\frac{5r_c^2}{4}}}=\frac{1}{\sqrt{5}}
\end{equation}
\begin{equation}\label{cosHb}
\cos(\pi-\alpha)=\cos\pi\cos\alpha+\sin\pi\sin\alpha=-\cos\alpha=-\frac{1}{\sqrt{5}}
\end{equation}


Using Eqs.~(\ref{sinH}), and (\ref{cosH}) in Eq.~(\ref{bbr}):
% $$B_r^{top}=\frac{\muo I 2}{2a\sqrt{5}}\sum_{n=1}^{\infty} \left(\frac{r}{a}\right)^{l-1}P_l^1\left(\frac{1}{\sqrt{5}}\right) P_l(\cos\theta)$$
% So,
\begin{equation}\label{bbrH}
B_r^{top}=\frac{\muo I }{\sqrt{5}a}\sum_{n=1}^{\infty} \left(\frac{r}{a}\right)^{l-1}P_l^1\left(\frac{1}{\sqrt{5}}\right) P_l(\cos\theta)
\end{equation}

Using Eqs.~(\ref{sinHb}), and (\ref{cosHb}) in Eq.~(\ref{bbr}):
\begin{equation}\label{bbrHb}
B_r^{bottom}=\frac{\muo I }{\sqrt{5}a}\sum_{n=1}^{\infty} \left(\frac{r}{a}\right)^{l-1}P_l^1\left(-\frac{1}{\sqrt{5}}\right) P_l(\cos\theta)
\end{equation}

\begin{equation}\label{brtot}
B_r=B_r^{top}+B_r^{bottom}
\end{equation}

\begin{multline}\label{brtot_1}
    B_r=\frac{\muo I }{\sqrt{5}a}\sum_{n=1}^{\infty} \left(\frac{r}{a}\right)^{l-1}P_l^1\left(\frac{1}{\sqrt{5}}\right) P_l(\cos\theta) \\+\frac{\muo I }{\sqrt{5}a}\sum_{n=1}^{\infty} \left(\frac{r}{a}\right)^{l-1}P_l^1\left(-\frac{1}{\sqrt{5}}\right) P_l(\cos\theta)
\end{multline}


\begin{equation}\label{leg}
P_l^1(-u)=(-1)^{l+1}P_l^1(u)
\end{equation}

From Eq.~(\ref{leg}) we get that only the odd n terms contribute to net field of Eq.~(\ref{brtot_1}). Furthermore, since \(P_3^1\left(\pm\frac{1}{\sqrt{5}}\right)\) is uniquely zero – thereby eliminating the n = 3.

\begin{equation}\label{brtotf}
B_r=\frac{2 \muo I }{\sqrt{5}a}\sum_{n=1,5,7,...}^{\infty} \left(\frac{r}{a}\right)^{l-1}P_l^1\left(\frac{1}{\sqrt{5}}\right) P_l(\cos\theta)
\end{equation}

Similarly, using Eqs.~(\ref{sinH}) and (\ref{cosH}) in Eq.~\ref{bbt}) :
% $$B_\theta^{top}=-\frac{\muo I 2}{2a\sqrt{5}}\sum_{n=1}^{\infty} \frac{1}{n}\left(\frac{r}{a}\right)^{l-1}P_l^1\left(\frac{1}{\sqrt{5}}\right) P_l^1(\cos\theta)$$
% So,
\begin{equation}\label{bbtH}
B_\theta^{top}=-\frac{\muo I }{\sqrt{5}a}\sum_{n=1}^{\infty}\frac{1}{n} \left(\frac{r}{a}\right)^{l-1}P_l^1\left(\frac{1}{\sqrt{5}}\right) P_l^1(\cos\theta)
\end{equation}

Again, using Eqs.~(\ref{sinHb}) and (\ref{cosHb}) in Eq.~\ref{bbt}) :
\begin{equation}\label{bbtHb}
B_\theta^{bottom}=-\frac{\muo I }{\sqrt{5}a}\sum_{n=1}^{\infty}\frac{1}{n} \left(\frac{r}{a}\right)^{l-1}P_l^1\left(\frac{1}{\sqrt{5}}\right) P_l^1(\cos\theta)
\end{equation}

\begin{equation}\label{bttot}
B_\theta=B_\theta^{top}+B_\theta^{bottom}
\end{equation}

\begin{multline}\label{bttot_1}
    B_\theta=-\frac{\muo I }{\sqrt{5}a}\sum_{n=1}^{\infty}\frac{1}{n} \left(\frac{r}{a}\right)^{l-1}P_l^1\left(\frac{1}{\sqrt{5}}\right) P_l^1(\cos\theta) \\-\frac{\muo I }{\sqrt{5}a}\sum_{n=1}^{\infty}\frac{1}{n} \left(\frac{r}{a}\right)^{l-1}P_l^1\left(\frac{1}{\sqrt{5}}\right) P_l^1(\cos\theta)
\end{multline}


As discussed above, Eq.~(\ref{bttot_1}) will be reduced to -
\begin{equation}\label{brtotf}
B_\theta=-\frac{2 \muo I }{\sqrt{5}a}\sum_{n=1,5,7,...}^{\infty}
\frac{1}{n}\left(\frac{r}{a}\right)^{l-1}P_l^1\left(\frac{1}{\sqrt{5}}\right) P_l^1(\cos\theta)
\end{equation}



Now again for Anti-Helmholtz coils of radius \(r_c\) located at \(z=\pm z_o\) i.e. for two loops at \(\theta=\alpha, \pi-\alpha \), \(I_2=-I_1=I\) and \(z_o=\sqrt{3}\frac{r_c}{2}\),
\begin{equation}\label{sinAH}
\sin\alpha=\frac{r_c}{a}=\frac{r_c}{\sqrt{z_o^2+r_c^2}}=\frac{r_c}{\sqrt{\left(\frac{\sqrt{3}r_c}{2}\right)^2+r_c^2}}=\frac{r_c}{\sqrt{\frac{7r_c^2}{4}}}=\frac{2}{\sqrt{7}}
\end{equation}
\begin{equation}\label{sinAHb}
\sin(\pi-\alpha)=\sin\pi\cos\alpha-\cos\pi\sin\alpha=\sin\alpha=\frac{2}{\sqrt{7}}
\end{equation}
\begin{equation}\label{cosAH}
\cos\alpha=\frac{z_o}{a}=\frac{\sqrt{3}r_c}{2\sqrt{z_o^2+r_c^2}}=\frac{\sqrt{3}r_c}{2\sqrt{\left(\frac{\sqrt{3}r_c}{2}\right)^2+r_c^2}}=\frac{\sqrt{3}r_c}{2\sqrt{\frac{7r_c^2}{4}}}=\sqrt{\frac{3}{7}}
\end{equation}
\begin{equation}\label{cosAHb}
\cos(\pi-\alpha)=\cos\pi\cos\alpha+\sin\pi\sin\alpha=-\cos\alpha=-\sqrt{\frac{3}{7}}
\end{equation}

Using Eqs.~(\ref{sinAH}) and (\ref{cosAH}) in Eq.~(\ref{bbr}) :
% $$B_{r(anti)}^{top}=\frac{\muo (-I) 2}{2a\sqrt{7}}\sum_{n=1}^{\infty} \left(\frac{r}{a}\right)^{l-1}P_l^1\left(\sqrt{\frac{3}{7}}\right) P_l(\cos\theta)$$
% So,
\begin{equation}\label{bbrAH}
B_{r(anti)}^{top}=-\frac{\muo I }{\sqrt{7}a}\sum_{n=1}^{\infty} \left(\frac{r}{a}\right)^{l-1}P_l^1\left(\sqrt{\frac{3}{7}}\right) P_l(\cos\theta)
\end{equation}

Using Eqs.~(\ref{sinAHb}) and (\ref{cosAHb}) in Eq.~(\ref{bbr}) :
\begin{equation}\label{bbrAHb}
B_{r(anti)}^{bottom}=\frac{\muo I }{\sqrt{7}a}\sum_{n=1}^{\infty} \left(\frac{r}{a}\right)^{l-1}P_l^1\left(-\sqrt{\frac{3}{7}}\right) P_l(\cos\theta)
\end{equation}

$$B_{r(anti)}=B_{r(anti)}^{top}+B_{r(anti)}^{bottom}$$

\begin{multline}\label{brtotAH_1}
    B_{r(anti)}=-\frac{\muo I }{\sqrt{7}a}\sum_{n=1}^{\infty} \left(\frac{r}{a}\right)^{l-1}P_l^1\left(\sqrt{\frac{3}{7}}\right) P_l(\cos\theta) \\+\frac{\muo I }{\sqrt{7}a}\sum_{n=1}^{\infty} \left(\frac{r}{a}\right)^{l-1}P_l^1\left(-\sqrt{\frac{3}{7}}\right) P_l(\cos\theta)
\end{multline}

From Eq.~(\ref{leg}) we get that only the even n terms contribute to net field of Eq.~(\ref{brtotAH_1}). So, we get,
\begin{equation}\label{brtotAH}
B_{r(anti)}=-\frac{2 \muo I }{\sqrt{7}a}\sum_{n=2,4,6,...}^{\infty}
\left(\frac{r}{a}\right)^{l-1}P_l^1\left(\sqrt{\frac{3}{7}}\right) P_l(\cos\theta)
\end{equation}


Similarly, using Eqs.~(\ref{sinAH}) and (\ref{cosAH}) in Eq.~(\ref{bbt}) :
% $$B_{\theta(anti)}^{top}=-\frac{\muo (-I) 2}{2a\sqrt{7}}\sum_{n=1}^{\infty} \frac{1}{n}\left(\frac{r}{a}\right)^{l-1}P_l^1\left(\sqrt{\frac{3}{7}}\right) P_l^1(\cos\theta)$$
% So,
\begin{equation}\label{bbtAH}
B_{\theta(anti)}^{top}=\frac{\muo I }{\sqrt{7}a}\sum_{n=1}^{\infty} \frac{1}{n}\left(\frac{r}{a}\right)^{l-1}P_l^1\left(\sqrt{\frac{3}{7}}\right) P_l^1(\cos\theta)
\end{equation}

Using Eqs.~(\ref{sinAHb}) and (\ref{cosAHb}) in Eq.~(\ref{bbt}) :
\begin{equation}\label{bbtAHb}
B_{\theta(anti)}^{bottom}=-\frac{\muo I }{\sqrt{7}a}\sum_{n=1}^{\infty}\frac{1}{n} \left(\frac{r}{a}\right)^{l-1}P_l^1\left(-\sqrt{\frac{3}{7}}\right) P_l^1(\cos\theta)
\end{equation}


$$B_{\theta(anti)}=B_{\theta(anti)}^{top}+B_{\theta(anti)}^{bottom}$$

\begin{multline}\label{bttotAH_1}
    B_{\theta(anti)}=\frac{\muo I }{\sqrt{7}a}\sum_{n=1}^{\infty} \frac{1}{n}\left(\frac{r}{a}\right)^{l-1}P_l^1\left(\sqrt{\frac{3}{7}}\right) P_l^1(\cos\theta) \\-\frac{\muo I }{\sqrt{7}a}\sum_{n=1}^{\infty}\frac{1}{n} \left(\frac{r}{a}\right)^{l-1}P_l^1\left(-\sqrt{\frac{3}{7}}\right) P_l^1(\cos\theta)
\end{multline}


As discussed above, Eq.~(\ref{bttotAH_1}) will be reduced to -
\begin{equation}\label{bttotAH}
B_{\theta(anti)}=\frac{2 \muo I }{\sqrt{7}a}\sum_{n=2,4,6,...}^{\infty}
\frac{1}{n}\left(\frac{r}{a}\right)^{l-1}P_l^1\left(\sqrt{\frac{3}{7}}\right) P_l^1(\cos\theta)
\end{equation}