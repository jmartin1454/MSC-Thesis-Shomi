
\chapter{Magnetic field inside the bulk of a mu-metal shield}
\lhead{\emph{Magnetic field inside the bulk of a mu-metal shield}}
% \newpage
% \label{chap:appendix_bustamante}
% \section{Magnetic field inside the bulk of a mu-metal shield}
\section{Analytic solution assuming linear permeability}
\subsection{General solution for an applied zonal field}
\subsubsection{Using Scalar Potential}
% \lhead{\emph{Magnetic field inside the bulk of a mu-metal shield}}

% \subsubsection{Using Scalar Potential}
Let us, place a spherical shell of thickness, t=b-a, which is made of material of permeability $\mu_r$, in a uniform magnetic field, \(\bm B_{\rm ext}=\bm{B_o}= \muo \bm H_o\) \cite{jackson}.

We know, \(\bm{\nabla}\times\bm{H}=\bm{J}\). But, we are considering no free current. So, \(\bm{J}_f=\bm{K}_f=0\). Thus, we get , \(\bm{\nabla}\times\bm{H}=0\) everywhere which implies that there exists a magnetic scalar potential \(\Phi_m\) that is everywhere continuous and
\begin{equation}\label{H}
\bm{H}=-\bm{\nabla}\Phi_m
\end{equation} 
Again, \(\bm{B}=\mu_r\bm{H}\) and \(\bm{\nabla}\cdot\bm{B}=0\). So, we get,
\begin{equation}\label{nablaH}
\bm{\nabla}\cdot\bm{H}=0
\end{equation}
Therefore, using eq. [\ref{H},\ref{nablaH}] we get,
\begin{equation}
\bm{\nabla}\cdot\bm{H}=-\bm{\nabla}\Phi_m^2=0
\end{equation}
So, $\Phi_m$ satisfies the Laplace equation and moreover, there exists azimuthal symmetry (i.e. m=0). Therefore, we can get,
\begin{equation}
\Phi(r,\theta)=\sum_{l=0}^{\infty} [A_lr^l+B_lr^{-(l+1)}]P_l(\cos\theta)
\end{equation}

For r\textgreater b we get,
\begin{equation}\label{r>b}
\Phi_m=-Hr^lP_l(\cos\theta)+\sum_{l=0}^{\infty}\frac{\alpha}{r^{l+1}}P_l(\cos\theta)
\end{equation}

For a\textless r\textless b we get,
\begin{equation}\label{a<r<b}
\Phi(r,\theta)=\sum_{l=0}^{\infty} \left(\beta r^l+\frac{\gamma}{r^{l+1}}\right)P_l(\cos\theta)
\end{equation}

And lastly, for r\textless a we get,
\begin{equation}\label{r<a}
\Phi(r,\theta)=\sum_{l=0}^{\infty} \delta r^l P_l(\cos\theta)
\end{equation}

So, the problem reduces to finding the proper solutions in the different regions to satisfy the following boundary conditions at r=a and r=b,


\begin{equation}\label{b1}   
\bm{H_2}\times n = \bm{H_1}\times n  
\end{equation}

\begin{equation}\label{b2}   
\bm{H_2}\cdot n=\frac{\mu_{r1}}{\mu_{r2}}\bm{H_1}\cdot n
\end{equation}

$$(\bm{B_2} - \bm{B_1})\times n = \muo \bm{K_b}$$
That's,
\begin{equation}\label{b3}   
(\mu_2\bm{H_2} - \mu_1\bm{H_1})\times n = \muo \bm{K_b}  
\end{equation}

Now using Eq.[\ref{H}] and boundary condition from Eq.[\ref{b1}] we get,
$$\frac{\delta \Phi_3}{\delta \theta}(b)=\frac{\delta \Phi_2}{\delta \theta}(b)$$

Using Eq.[\ref{r>b}, \ref{a<r<b}]
$$-Hb^l\frac{\delta}{\delta \theta}(P_l(\cos\theta))+\frac{\alpha}{r^{l+1}}\frac{\delta}{\delta \theta}(P_l(\cos\theta))=\left(\beta b^l+\frac{\gamma}{b^{l+1}}\right)\frac{\delta}{\delta \theta}(P_l(\cos\theta))$$
\begin{equation}\label{e01}
-Hb^l+\frac{\alpha}{b^{l+1}}=\beta b^l+\frac{\gamma}{b^{l+1}}
\end{equation}

Similarly,
$$\frac{\delta \Phi_2}{\delta \theta}(a)=\frac{\delta \Phi_1}{\delta \theta}(a)$$
Using Eq.[\ref{a<r<b}, \ref{r<a}]
$$\left(\beta a^l+\frac{\gamma}{a^{l+1}}\right)\frac{\delta}{\delta \theta}(P_l(\cos\theta))=\delta a^l \frac{\delta}{\delta \theta}(P_l(\cos\theta))$$
\begin{equation}\label{e02}
\beta a^l+\frac{\gamma}{a^{l+1}}=\delta a^l
\end{equation}

Similarly, using Eq.[\ref{H}] and another boundary condition from Eq.[\ref{b2}] we get,
$$\muo\frac{\delta \Phi_3}{\delta r}(b)=\mu_r\frac{\delta \Phi_2}{\delta r}(b)$$
Using Eq.[\ref{r>b}, \ref{a<r<b}]
$$\muo\left[-Hlb^{l-1}+\frac{-(l+1)\alpha}{b^{l+2}}\right]P_l(\cos\theta)=\mu_r\left[\beta l b^{l-1}+\frac{-(l+1)\gamma}{b^{l+2}}\right]P_l(\cos\theta)$$
\begin{equation}\label{e03}
-Hlb^{l-1}-\frac{(l+1)\alpha}{b^{l+2}}=\frac{\mu_r}{\muo}\left[\beta l b^{l-1}-\frac{(l+1)\gamma}{b^{l+2}}\right]
\end{equation}
Again,
$$\mu_r\frac{\delta \Phi_2}{\delta r}(a)=\muo\frac{\delta \Phi_1}{\delta r}(a)$$
Using Eq.[\ref{a<r<b}, \ref{r<a}]
$$\frac{\mu_r}{\muo}\left[\beta l a^{l-1}+\frac{-(l+1)\gamma}{a^{l+2}}\right]P_l(\cos\theta)=\delta l a^{l-1} P_l(\cos\theta)$$
\begin{equation}\label{e04}
\delta l a^{l-1}=\mu_r'\left[\beta l a^{l-1}-\frac{(l+1)\gamma}{a^{l+2}}\right]
\end{equation}

From eq. [\ref{e01}] we get,
$$-Hb^l+\frac{\alpha}{b^{l+1}}=\beta b^l+\frac{\gamma}{b^{l+1}}$$
$$\frac{-Hb^{2l+1}+\alpha}{b^{l+1}}=\frac{\beta b^{2l+1}+\gamma}{b^{l+1}}$$
$$-Hb^{2l+1}+\alpha=\beta b^{2l+1}+\gamma$$
\begin{equation}
\alpha- b^{2l+1}\beta-\gamma=b^{2l+1}H
\end{equation}

From eq.[\ref{e02}] we get,
$$\beta a^l+\frac{\gamma}{a^{l+1}}=\delta a^l$$
$$\frac{\beta a^{2l+1}+\gamma}{a^{l+1}}=\delta a^l$$
\begin{equation}
 a^{2l+1}\beta+\gamma-a^{2l+1}\delta =0
\end{equation}

From eq. [\ref{e03}] we get,
$$-Hlb^{l-1}-\frac{(l+1)\alpha}{b^{l+2}}=\mu_r'\left[\beta l b^{l-1}-\frac{(l+1)\gamma}{b^{l+2}}\right]$$
$$\frac{-Hlb^{2l+1}-(l+1)\alpha}{b^{l+2}}=\mu_r'\left[\frac{\beta l b^{2l+1}-(l+1)\gamma}{b^{l+2}}\right]$$
\begin{equation}
(l+1)\alpha+\mu_r' l b^{2l+1}\beta-\mu_r'(l+1)\gamma=-lb^{2l+1}H
\end{equation}

From eq.[\ref{e04}] we get,
$$\delta l a^{l-1}=\mu_r'\left[\beta l a^{l-1}-\frac{(l+1)\gamma}{a^{l+2}}\right]$$
$$\delta l a^{l-1}=\mu_r'\left[\frac{\beta l a^{2l+1}-(l+1)\gamma}{a^{l+2}}\right]$$
$$\delta l a^{2l+1}=\mu_r'\beta l a^{2l+1}-\mu_r'(l+1)\gamma$$
\begin{equation}
\mu_r'l a^{2l+1}\beta -\mu_r'(l+1)\gamma- l a^{2l+1}\delta=0
\end{equation}

So we get,
\begin{equation}\label{e1}
\alpha- b^{2l+1}\beta-\gamma=b^{2l+1}H
\end{equation}
\begin{equation}\label{e2}
 a^{2l+1}\beta+\gamma-a^{2l+1}\delta =0
\end{equation}
\begin{equation}\label{e3}
(l+1)\alpha+\mu_r' l b^{2l+1}\beta-\mu_r'(l+1)\gamma=-lb^{2l+1}H
\end{equation}
\begin{equation}\label{e4}
\mu_r'l a^{2l+1}\beta -\mu_r'(l+1)\gamma- l a^{2l+1}\delta=0
\end{equation}

Subtracting eq. [\ref{e2}] from eq.[\ref{e4}]/$l$ we get,
$$\mu_r' a^{2l+1}\beta -\left(\frac{l+1}{l}\right)\mu_r'\gamma- a^{2l+1}\delta-a^{2l+1}\beta-\gamma+a^{2l+1}\delta=0$$
$$ \beta a^{2l+1}(\mu_r'-1) -\gamma\left[\left(\frac{l+1}{l}\right)\mu_r'+1\right]=0$$
So we get,
\begin{equation}\label{g1}
\gamma=\beta \left[\frac{a^{2l+1}(\mu_r'-1)}{\left(\frac{l+1}{l}\right)\mu_r'+1}\right]
\end{equation}

Again, subtracting eq. [\ref{e1}]$*(l+1)$ from eq.[\ref{e3}] we get,
$$(l+1)\alpha+\mu_r' l b^{2l+1}\beta-\mu_r'(l+1)\gamma+lb^{2l+1}H-(l+1)\alpha+(l+1) b^{2l+1}\beta+(l+1)\gamma+(l+1)b^{2l+1}H$$
$$(\mu_r'l+l+1) b^{2l+1}\beta+(1-\mu_r')(l+1)\gamma=-(l+1+l)b^{2l+1}H$$
Putting the value of eq.[\ref{g1}] we get,
$$(\mu_r'l+l+1) b^{2l+1}\beta+(l+1)a^{2l+1}\beta \left[\frac{(1-\mu_r')(\mu_r'-1)}{\left(\frac{l+1}{l}\right)\mu_r'+1}\right]=-(2l+1)b^{2l+1}H$$
$$\beta(\mu_r'l+l+1) -\beta(l+1)\left(\frac{a}{b}\right)^{2l+1} \left[\frac{(\mu_r'-1)^2}{\left(\frac{l+1}{l}\right)\mu_r'+1}\right]=-(2l+1)H$$
$$\beta\left[(\mu_r'l+l+1) -(l+1)\left(\frac{a}{b}\right)^{2l+1} \left[\frac{(\mu_r'-1)^2}{\left(\frac{l+1}{l}\right)\mu_r'+1}\right]\right]=-(2l+1)H$$
$$\beta\left[\frac{(\mu_r'l+l+1)\left[\left(\frac{l+1}{l}\right)\mu_r'+1\right]-(l+1)\left(\frac{a}{b}\right)^{2l+1}(\mu_r'-1)^2}{\left(\frac{l+1}{l}\right)\mu_r'+1}\right]=-(2l+1)H$$
So we get,
\begin{equation}
\beta=-\left[\frac{(2l+1)\left(\left(\frac{l+1}{l}\right)\mu_r'+1\right)}{(\mu_r'l+l+1)\left[\left(\frac{l+1}{l}\right)\mu_r'+1\right]-(l+1)\left(\frac{a}{b}\right)^{2l+1}(\mu_r'-1)^2}\right]H
\end{equation}

For \(\mu_r\gg\muo \Rightarrow \mu_r/\muo\gg 1 \Rightarrow \mu_r'\gg 1\) we get,
$$\beta\approx-\left[\frac{\left[\frac{(2l+1)(l+1)}{l}\right]\mu_r'}{(l+1)\mu_r'^2-(l+1)\left(\frac{a}{b}\right)^{2l+1}\mu_r'^2}\right]H$$
$$\beta\approx-\left[\frac{(2l+1)(l+1)\mu_r'}{l(l+1)\mu_r'^2\left[1-\left(\frac{a}{b}\right)^{2l+1}}\right]\right]H$$
So we get,
\begin{equation}\label{bt2}
\beta\approx-\left[\frac{(2l+1)\muo}{l\mu_r\left[1-\left(\frac{a}{b}\right)^{2l+1}}\right]\right]H
\end{equation}

Putting the value of Eq.[\ref{b1}] in Eq.[\ref{g1}] we get,
$$\gamma=-\left[\frac{(2l+1)\left(\left(\frac{l+1}{l}\right)\mu_r'+1\right)}{(\mu_r'l+l+1)\left[\left(\frac{l+1}{l}\right)\mu_r'+1\right]-(l+1)\left(\frac{a}{b}\right)^{2l+1}(\mu_r'-1)^2}\right] \left[\frac{a^{2l+1}(\mu_r'-1)}{\left(\frac{l+1}{l}\right)\mu_r'+1}\right]H$$
\begin{equation}
\gamma=-\left[\frac{(2l+1)a^{2l+1}(\mu_r'-1)}{(\mu_r'l+l+1)\left[\left(\frac{l+1}{l}\right)\mu_r'+1\right]-(l+1)\left(\frac{a}{b}\right)^{2l+1}(\mu_r'-1)^2}\right] H    
\end{equation}


So, for \(\mu_r'\gg 1\) we get,
$$\gamma\approx-\left[\frac{(2l+1)a^{2l+1}\mu_r'}{\mu_r'^2(l+1)-(l+1)\left(\frac{a}{b}\right)^{2l+1}\mu_r'^2}\right] H$$
$$\gamma\approx-\left[\frac{(2l+1)a^{2l+1}\mu_r'}{\mu_r'^2(l+1)\left(1-\left(\frac{a}{b}\right)^{2l+1}\right)}\right] H$$
So, we get,
\begin{equation}\label{g2}
\gamma\approx-\left[\frac{(2l+1)a^{2l+1}\muo}{\mu_r(l+1)\left(1-\left(\frac{a}{b}\right)^{2l+1}\right)}\right] H
\end{equation}
From Eq. [\ref{e2}] we get,
$$\delta =\beta+\frac{\gamma}{a^{2l+1}}$$
$$\delta =-\left[\frac{1}{(\mu_r'l+l+1)\left[\left(\frac{l+1}{l}\right)\mu_r'+1\right]-(l+1)\left(\frac{a}{b}\right)^{2l+1}(\mu_r'-1)^2}\right]$$ $$\left[(2l+1)\left(\left(\frac{l+1}{l}\right)\mu_r'+1\right)+(2l+1)(\mu_r'-1)\right]H$$
$$\delta =-\left[\frac{(2l+1)\left[\left(\frac{l+1}{l}\right)\mu_r'+1+\mu_r'-1\right]}{(\mu_r'l+l+1)\left[\left(\frac{l+1}{l}\right)\mu_r'+1\right]-(l+1)\left(\frac{a}{b}\right)^{2l+1}(\mu_r'-1)^2}\right]H$$
$$\delta =-\left[\frac{(2l+1)\left[\mu_r'\left(\frac{l+1}{l}+1\right)\right]}{(\mu_r'l+l+1)\left[\left(\frac{l+1}{l}\right)\mu_r'+1\right]-(l+1)\left(\frac{a}{b}\right)^{2l+1}(\mu_r'-1)^2}\right]H$$
So we get,
\begin{equation}
\delta =-\left[\frac{\frac{(2l+1)^2}{l}\mu_r'}{(\mu_r'l+l+1)\left[\left(\frac{l+1}{l}\right)\mu_r'+1\right]-(l+1)\left(\frac{a}{b}\right)^{2l+1}(\mu_r'-1)^2}\right]H    
\end{equation}


Therefore,

\begin{equation}\label{bt1}
\beta=-\left[\frac{(2l+1)\left(\left(\frac{l+1}{l}\right)\mu_r'+1\right)}{(\mu_r'l+l+1)\left[\left(\frac{l+1}{l}\right)\mu_r'+1\right]-(l+1)\left(\frac{a}{b}\right)^{2l+1}(\mu_r'-1)^2}\right]H
\end{equation}
\begin{equation}\label{g1}
\gamma=-\left[\frac{(2l+1)a^{2l+1}(\mu_r'-1)}{(\mu_r'l+l+1)\left[\left(\frac{l+1}{l}\right)\mu_r'+1\right]-(l+1)\left(\frac{a}{b}\right)^{2l+1}(\mu_r'-1)^2}\right] H    
\end{equation}
\begin{equation}\label{dt1}
\delta =-\left[\frac{\frac{(2l+1)^2}{l}\mu_r'}{(\mu_r'l+l+1)\left[\left(\frac{l+1}{l}\right)\mu_r'+1\right]-(l+1)\left(\frac{a}{b}\right)^{2l+1}(\mu_r'-1)^2}\right]H    
\end{equation}
Now using Eq.[\ref{H}] and boundary condition from Eq.[\ref{b3}] we get,
$$-\mu_r \frac{1}{r}\frac{\delta \Phi_2}{\delta \theta}(a)+\muo \frac{1}{r}\frac{\delta \Phi_1}{\delta \theta}(a)=\muo\bm{K_1}  $$
$$\frac{1}{r}\frac{\delta \Phi_1}{\delta \theta}(a)-\frac{1}{r}\mu'_r\frac{\delta \Phi_2}{\delta \theta}(a)=\bm{K_1}  $$
Using Eq.[\ref{a<r<b}, \ref{r<a}]
$$\frac{1}{a}\left[\delta a^l-\mu'_r\left(\beta a^l+\frac{\gamma}{a^{l+1}}\right)\right]\frac{\delta}{\delta \theta}(P_l(\cos\theta))= \bm{K_1}$$
$$\frac{1}{a}\left[\delta a^l-\mu'_r\left(\beta a^l+\frac{\gamma}{a^{l+1}}\right)\right][-P_l^1(u)]= K_1 P_l^1(u)$$
$$a K_1=\mu'_r\left(\beta a^l+\frac{\gamma}{a^{l+1}}\right)\right]-\delta a^l$$
$$\frac{(2l+1)a^{l-1+1}\mathcal{K}_1}{\muo}=\mu'_r\left(\beta a^l+\frac{\gamma}{a^{l+1}}\right)\right]-\delta a^l$$
$$\frac{(2l+1)a^l\mathcal{K}_1}{\muo}=a^l\left[\mu'_r\left(\beta +\frac{\gamma}{a^{2l+1}}\right)-\delta \right]$$
Using Eq. [\ref{bt1}, \ref{g1}, \ref{dt1}] we get,
$$\frac{(2l+1)\mathcal{K}_1}{\muo}=-\left[\frac{2l+1}{(\mu_r'l+l+1)\left[\left(\frac{l+1}{l}\right)\mu_r'+1\right]-(l+1)\left(\frac{a}{b}\right)^{2l+1}(\mu_r'-1)^2}\right]$$ $$\left[\mu'_r\left(\left(\frac{l+1}{l}\right)\mu_r'+1+\mu_r'-1\right)-\frac{(2l+1)}{l}\mu_r'}\right]H$$
$$\frac{\mathcal{K}_1}{\muo}=-\left[\frac{\mu'_r\left(\left(\frac{l+1}{l}\right)\mu_r'+\mu_r'-\frac{2l+1}{l}\right)}{(\mu_r'l+l+1)\left[\left(\frac{l+1}{l}\right)\mu_r'+1\right]-(l+1)\left(\frac{a}{b}\right)^{2l+1}(\mu_r'-1)^2}\right]H$$
$$\frac{\mathcal{K}_1}{\muo}=-\left[\frac{\mu'_r\left(\frac{\mu_r'l+\mu_r'+\mu_r'l-2l-1}{l}\right)}{(\mu_r'l+l+1)\left[\left(\frac{l+1}{l}\right)\mu_r'+1\right]-(l+1)\left(\frac{a}{b}\right)^{2l+1}(\mu_r'-1)^2}\right]H$$
$$\frac{\mathcal{K}_1}{\muo}=-\frac{1}{l}\left[\frac{\mu'_r\left(\mu_r'(2l+1)-1(2l+1)\right)}{(\mu_r'l+l+1)\left[\left(\frac{l+1}{l}\right)\mu_r'+1\right]-(l+1)\left(\frac{a}{b}\right)^{2l+1}(\mu_r'-1)^2}\right]H$$
So we get,
\begin{equation}\label{k1s}
\mathcal{K}_1=-\frac{\muo H}{l}\left[\frac{(2l+1)(\mu_r'-1)\mu'_r}{(\mu_r'l+l+1)\left[\left(\frac{l+1}{l}\right)\mu_r'+1\right]-(l+1)\left(\frac{a}{b}\right)^{2l+1}(\mu_r'-1)^2}\right]
\end{equation}
From bound surface currents method we got,
$$\mathcal{K}_1=-\frac{G_l(2l+1)(l+1)(\mu_r-\muo)\mu_r}{[\mu_r(l+1)+\muo l)][\mu_r l+\muo(l+1)]-l(l+1)\left(\dfrac{a}{b}\right)^{2l+1}(\mu_r-\muo)^2}$$
$$\mathcal{K}_1=-\frac{G_l(2l+1)(l+1)(\mu_r'-1)\muo\mu_r}{[\muo(\mu_r' l+\mu_r'+ l)\muo(\mu_r' l+l+1)]-l(l+1)\left(\dfrac{a}{b}\right)^{2l+1}\muo^2(\mu_r'-1)^2}$$
$$\mathcal{K}_1=-\frac{G_l(2l+1)(l+1)(\mu_r'-1)\mu_r'}{[(\mu_r' l+\mu_r'+ l)l(\mu_r'( 1+\frac{1}{l})+1)]-l(l+1)\left(\dfrac{a}{b}\right)^{2l+1}(\mu_r'-1)^2}$$
So we get,
\begin{equation}\label{k1b}
\mathcal{K}_1=-\frac{G_l (l+1)}{l}\left[\frac{(2l+1)(\mu_r'-1)\mu'_r}{(\mu_r'l+l+1)\left[\left(\frac{l+1}{l}\right)\mu_r'+1\right]-(l+1)\left(\frac{a}{b}\right)^{2l+1}(\mu_r'-1)^2}\right]
\end{equation}

So, comparing Eq.[\ref{k1s} and \ref{k1b}]
\begin{equation}\label{g}
G_l(l+1)=\muo H
\end{equation}

So, if Eq. [\ref{g}] is correct, then we are getting same result using scalar potential as we got from bound surface current method for \mathcal{K}_1.




So for \(a\textless r\textless b \) from Eq. [\ref{a<r<b}],

$$\Phi_m=\left(\beta r^l+\frac{\gamma}{r^{l+1}}\right)P_l(\cos\theta)$$
$$\Phi_m=\left[-r^l\left[\frac{(2l+1)\muo}{l\mu_r\left[1-\left(\frac{a}{b}\right)^{2l+1}}\right]\right]H-\frac{r^l}{r^{2l+1}}\left[\frac{(2l+1)a^{2l+1}\muo}{\mu_r(l+1)\left(1-\left(\frac{a}{b}\right)^{2l+1}\right)}\right] H}\right]P_l(\cos\theta)$$
$$\Phi_m=-\frac{(2l+1)Hr^lP_l(\cos\theta)}{\mu_r\left[1-\left(\frac{a}{b}\right)^{2l+1}}\right]}\left[\frac{\muo}{l}+\left(\frac{a}{r}\right)^{2l+1}\left(\frac{\muo}{l+1}\right)\right]$$
$$\Phi_m=-\frac{(2l+1)(\muo H)r^lP_l(\cos\theta)}{\mu_r(l+1)}\left[\frac{\frac{l+1}{l}+\left(\frac{a}{r}\right)^{2l+1}}{1-\left(\frac{a}{b}\right)^{2l+1}}\right]$$
$$\Phi_m=-\frac{(2l+1)Br^lP_l(\cos\theta)}{\mu_r(l+1)}\left[\frac{\frac{l+1}{l}+\left(\frac{a}{r}\right)^{2l+1}}{1-\left(\frac{a}{b}\right)^{2l+1}}\right]$$
\begin{equation}\label{p}
\Phi_m=-\frac{(2l+1)Br^lP_l(\cos\theta)}{\mu_r l}\left[\frac{1+\frac{l}{l+1}\left(\frac{a}{r}\right)^{2l+1}}{1-\left(\frac{a}{b}\right)^{2l+1}}\right]
\end{equation}
From Eq.[\ref{H}] we know,
$$\bm{H}=-\bm{\nabla}\Phi_m$$
$$H_r=-\frac{\delta \Phi_m}{\delta r}$$
Using eq. [\ref{p}] we get,
$$H_r=-\frac{\delta}{\delta r}\left[-\frac{(2l+1)Br^lP_l(\cos\theta)}{\mu_r l}\left[\frac{1+\frac{l}{l+1}\left(\frac{a}{r}\right)^{2l+1}}{1-\left(\frac{a}{b}\right)^{2l+1}}\right]\right]$$
$$H_r=\frac{(2l+1)BP_l(u)}{\mu_r l\left[1-\left(\frac{a}{b}\right)^{2l+1}\right]}\frac{\delta}{\delta r}\left[r^l+\frac{l}{l+1}a^{2l+1}r^{-2l-1+l}\right]$$
$$H_r=\frac{(2l+1)BP_l(u)}{\mu_r l\left[1-\left(\frac{a}{b}\right)^{2l+1}\right]}\left[l r^{l-1}+\frac{l}{l+1}a^{2l+1}\frac{(-l-1)}{r^{l+2}}\right]$$
$$H_r=\frac{(2l+1)BP_l(u)l r^{l-1}}{\mu_r l\left[1-\left(\frac{a}{b}\right)^{2l+1}\right]}\left[1-\left(\frac{a}{r}\right)^{2l+1}\right]$$
So we get,
\begin{equation}\label{Hr}
H_r=\frac{(2l+1)}{\mu_r}r^{l-1}BP_l(u)\left[\frac{1-\left(\frac{a}{r}\right)^{2l+1}}{1-\left(\frac{a}{b}\right)^{2l+1}}\right]
\end{equation}
Therefore,
\begin{equation}\label{Br}
B_r=(2l+1)r^{l-1}BP_l(u)\left[\frac{1-\left(\frac{a}{r}\right)^{2l+1}}{1-\left(\frac{a}{b}\right)^{2l+1}}\right]
\end{equation}

Similarly,
\begin{equation*}
\begin{split}
H_\theta &=-\frac{1}{r}\frac{\delta \Phi_m}{\delta \theta}\\
        & =-\frac{1}{r}\frac{\delta}{\delta \theta}\left[-\frac{(2l+1)Br^lP_l(\cos\theta)}{\mu_r l}\left[\frac{1+\frac{l}{l+1}\left(\frac{a}{r}\right)^{2l+1}}{1-\left(\frac{a}{b}\right)^{2l+1}}\right]\right]
\end{split}
\end{equation*}

So we get,
\begin{equation}\label{Ht}
H_\theta=-\frac{(2l+1)}{\mu_r l}Br^{l-1}P_l^1(u)}\left[\frac{1+\frac{l}{l+1}\left(\frac{a}{r}\right)^{2l+1}}{1-\left(\frac{a}{b}\right)^{2l+1}}\right]
\end{equation}

Therefore,
\begin{equation}\label{Bt}
B_\theta=-\frac{(2l+1)}{l}Br^{l-1}P_l^1(u)}\left[\frac{1+\frac{l}{l+1}\left(\frac{a}{r}\right)^{2l+1}}{1-\left(\frac{a}{b}\right)^{2l+1}}\right]
\end{equation}

So, finally,
\begin{equation}\label{B}
\bm{B}=(2l+1)r^{l-1}B_o P_l(u)\left[\frac{1-\left(\frac{a}{r}\right)^{2l+1}}{1-\left(\frac{a}{b}\right)^{2l+1}}\right] \hat{r}-\frac{(2l+1)}{l}B_o r^{l-1}P_l^1(u)}\left[\frac{1+\frac{l}{l+1}\left(\frac{a}{r}\right)^{2l+1}}{1-\left(\frac{a}{b}\right)^{2l+1}}\right] \hat{\theta}
\end{equation}





























\subsubsection{Using Equivalent Bound Surface Currents}

In general, any surface current bound to a sphere , and its resulting magnetic field, can be written in terms of spherical harmonics of order $m$ and degree $n$~\cite{CB1, smythe}.One can show , however, that the resulting equations arising from the boundary conditions on the tangential components of the magnetic field (i.e., $B$_\theta\  and \ $B$_\phi\)) are independent of the order $m$ of the spherical harmonic. Without loss of generality, then, we can restrict the analysis of spherical shields to zonal surface currents and field only (i.e. \phi-independent\ , \ $m$=0\) ).This also means that the following results can be applied to cases where tesseral components ($m$ \textgreater\) 0) do exist in the fields and currents, which is extremely valuable from the point of view of coil design, where the general spherical harmonics can be used as $building$ $blocks$ to produce a desire magnetic field.

From ~\cite{CB1, smythe}, the zonal surface current
\begin{equation}\label{i}
I_\phi\)=\bm{K}=KP_n^1(u)  \, \bm{\hat{\phi}}\)
\end{equation}
bound to a spherical surface $r=a$ gives rise to the vector potential
\begin{equation}\label{a}
\bm A =\mathcal{K} 
\left \{
  \begin{tabular}{ccc}
  r^n \bm{\hat{\phi}}\) &  & r\textless\) a  \\
  \(\frac{a^{2n+1}}{r^{n+1}}
\) &  & r\textgreater\) a  
  \end{tabular}
\right 
\end{equation}




where $P_n^1(u)$ is the associated Legendre function of order 1 and degree \($n$, $u$=cos\theta\), and the coefficient \(\mathcal{K}={\muo K}/({(2n+1)a^{n-1}})
\)has units T/m$^{n-1}$.

Consider a spherical $\mu_r$-metal shield of inner radius $r_1=R$ and outer radius $r_2=R+t$, and permeability \(\mu_r\) centered on the origin and exposed to the general zonal field (\textit{i.e.,} axisymmetric or $\phi$-independent) of order $n$~\cite{CB1, smythe}  in the presence of an externally applied magnetic field. 
\begin{equation}\label{bo}
\bm B_{\rm ext} = \bm{B_o} = G_n \, r^{n-1} \, (n+1)[n P_n(u) \, \bm{\hat{r}} -  P_n^1(u)  \, \bm{\hat{\theta}} ] \, ,
\label{BextS}
\end{equation}
where the magnitude $G_n$ is in units of T/m$^{n-1}$.
The magnetic fields arising from Eq. (\ref{a}) are
\begin{equation}\label{r=R}
\lim_{r\to R}\bm B =\mathcal{K}_1
\left \{
  \begin{tabular}{ccc}
  r^{n-1} \, (n+1)[n P_n(u) \, \bm{\hat{r}} -  P_n^1(u)  \, \bm{\hat{\theta}} ] \, &  & r\textless\) R  \\
  \(\frac{R^{2n+1}}{r^{n+2}}
\)n[(n+1) P_n(u) \, \bm{\hat{r}} +  P_n^1(u)  \, \bm{\hat{\theta}} ] \, &  & r\textgreater\) R  
  \end{tabular}
\right 
\end{equation}

\begin{equation}\label{r=R+t}
\lim_{r\to R+t}\bm B =\mathcal{K}_2
\left \{
  \begin{tabular}{ccc}
  r^{n-1} \, (n+1)[n P_n(u) \, \bm{\hat{r}} -  P_n^1(u)  \, \bm{\hat{\theta}} ] \, &  & r\textless\) R+t  \\
  \(\frac{{R+t}^{2n+1}}{r^{n+2}}
\)n[(n+1) P_n(u) \, \bm{\hat{r}} +  P_n^1(u)  \, \bm{\hat{\theta}} ] \, &  & r\textgreater\) R+t  
  \end{tabular}
\right \
\end{equation}
 The net field inside the shield (\textit{i.e.},  $r<R$) is 
\begin{equation}\label{r<R}
\lim_{r\to R}\bm B = (\mathcal{K}_1+\mathcal{K}_2+G_n) \, (n+1)\, r^{n-1} \, [n P_n(u) \, \bm{\hat{r}} -  P_n^1(u)  \, \bm{\hat{\theta}}
\end{equation}

The net field  within the bulk of the shield (\textit{i.e.}, $R<r<R+t$)  is 
\begin{equation}\label{R<r<R+t}
\lim_{r\to R<r<R+t}\bm B= \mathcal{K}_1 \, n\, \frac{R^{2n+1}}{r^{n+2}} \,[(n+1) P_n(u) \, \bm{\hat{r}} +  P_n^1(u)  \, \bm{\hat{\theta}} ] \,  +\, (\mathcal{K}_2+G_n) \, (n+1)\, r^{n-1} \, [n P_n(u) \, \bm{\hat{r}} -  P_n^1(u)  \, \bm{\hat{\theta}} ]
\end{equation}

The net field  outside the shield (\textit{i.e.}, $r>R+t$) is
\begin{multline}\label{r>R}
\lim_{r\to r>R+t}\bm B=\frac{\mathcal{K}_1R^{2n+1} + \mathcal{K}_2 (R+t)^{2n+1} }{r^{n+2}} \, n\, [(n+1) P_n(u) \, \bm{\hat{r}} +  P_n^1(u)  \, \bm{\hat{\theta}} ] \,  \\+\, G_n \, (n+1)\, r^{n-1} \, [n P_n(u) \, \bm{\hat{r}} - P_n^1(u)  \, \bm{\hat{\theta}} ]
\end{multline}

To find \(\mathcal{K}_1\) and \(\mathcal{K}_2\) , we have to apply the boundary condition for the tangential component of the magnetic field i.e.
\begin{equation}\label{bk}
\frac{1}{\muo} B_{1\theta}=  \frac{1}{\mu_r} B_{2\theta}  
\end{equation}

From Eq.  (\ref{r<R}), (\ref{R<r<R+t}) and (\ref{bk}) we get,
$$\(\mu_r B_{1\theta}=\muo B_{2\theta}\)$$
$$\(\mu_r[(\mathcal{K}_1+\mathcal{K}_2+G_n) \, (n+1)\, r^{n-1}(-  P_n^1(u))\ ]=\muo[\mathcal{K}_1 \, n\, \frac{R^{2n+1}}{r^{n+2}}P_n^1(u) \,  +\, (\mathcal{K}_2+G_n) \, (n+1)\, r^{n-1}( -  P_n^1(u))]\)$$

\begin{multline*}
-\mu_r \mathcal{K}_1 r^{n-1}(n+1) P_n^1(u)-\mu_r (\mathcal{K}_2+G_n)r^{n-1}(n+1) P_n^1(u)\\=\muo \mathcal{K}_1 n \frac{R^{2n+1}}{r^{n+2}}P_n^1(u)-\muo (\mathcal{K}_2+G_n)r^{n-1}(n+1) P_n^1(u)
\end{multline*}

$$\([\mu_r r^{n-1}(n+1)+\muo n \frac{R^{2n+1}}{r^{n+2}}]P_n^1(u)\mathcal{K}_1+(\mu_r-\muo)r^{n-1}(n+1)P_n^1(u)\mathcal{K}_2=-(\mu_r-\muo)r^{n-1}(n+1)P_n^1(u)G_n\)$$
Excluding $r^{n-1}(n+1)P_n^1(u)$ from both sides we get,
$$\([\mu_r+\muo\frac{n}{n+1}\left(\dfrac{R}{r}\right)^{2n+1}]\mathcal{K}_1 +(\mu_r-\muo)\mathcal{K}_2=-(\mu_r-\muo)G_n\)$$
So at r=R we get,
\begin{equation}\label{es1}
[\mu_r+\muo\frac{n}{n+1}]\mathcal{K}_1 +(\mu_r-\muo)\mathcal{K}_2=-(\mu_r-\muo)G_n
\end{equation}

Again, from Eq. (\ref{bk}) we get,
$$\(\frac{1}{\mu_r} B_{2\theta}=\frac{1}{\muo} B_{3\theta}\)$$
$$\(\muo B_{2\theta}=\mu_r B_{3\theta}\)$$

Using eq. , (\ref{R<r<R+t}), (\ref{r>R}) we get,
\begin{multline*}
\muo\left[\mathcal{K}_1 n\left(\frac{R^{2n+1}}{r^{n+2}}\right)P_n^1(u)+ (\mathcal{K}_2+G_n)(n+1)r^{n-1}\left(-P_n^1(u)\right)\right]\\= \mu_r\left[\frac{\mathcal{K}_1R^{2n+1} + \mathcal{K}_2 (R+t)^{2n+1} }{r^{n+2}}n P_n^1(u) +G_n (n+1) r^{n-1}\left(-  P_n^1(u)\right)\right ]
\end{multline*}

\begin{multline*}
(\mu_r-\muo)n\left(\frac{R^{2n+1}}{r^{n+2}}\right)P_n^1(u)\mathcal{K}_1+\left[\mu_r n\frac{(R+t)^{2n+1}}{r^{n+2}}+\muo(n+1)r^{n-1}\right]P_n^1(u)\mathcal{K}_2 \\= (\mu_r-\muo)r^{n-1}(n+1)P_n^1(u)G_n
\end{multline*}
Excluding $r^{n-1}(n+1)P_n^1(u)$ from both sides we get,
$$\((\mu_r-\muo)\frac{n}{n+1}\left(\dfrac{R}{r}\right)^{2n+1}\mathcal{K}_1+[\mu_r\frac{n}{n+1}\left(\dfrac{R+t}{r}\right)^{2n+1}+\muo]\mathcal{K}_2=(\mu_r-\muo)G_n\)$$
At $r=R+t$ we get,
\begin{equation}\label{es2}
(\mu_r-\muo)\frac{n}{n+1}\left(\dfrac{R}{R+t}\right)^{2n+1}\mathcal{K}_1+[\mu_r\frac{n}{n+1}+\muo]\mathcal{K}_2=(\mu_r-\muo)G_n
\end{equation}
Again from Eq. (\ref{es1}) we get,
$$\(\frac{[\mu_r+\muo\frac{n}{n+1}]}{(\mu_r-\muo)}\mathcal{K}_1 +\mathcal{K}_2=-G_n\)$$
\begin{equation}\label{k2}
\mathcal{K}_2=-G_n-\frac{\mu_r(n+1)+\muo n}{(\mu_r-\muo)(n+1)}\mathcal{K}_1
\end{equation}
Adding Eq. (\ref{es1}) and (\ref{es2}) we get,
\begin{multline*}
\left[\mu_r+\muo\frac{n}{n+1}\right]\mathcal{K}_1 +(\mu_r-\muo)\mathcal{K}_2+(\mu_r-\muo)G_n+(\mu_r-\muo)\frac{n}{n+1}\left(\dfrac{R}{R+t}\right)^{2n+1}\mathcal{K}_1\\+\left[\mu_r\frac{n}{n+1}+\muo\right]\mathcal{K}_2-(\mu_r-\muo)G_n=0
\end{multline*}
\begin{multline*}
[\mu_r+\muo\frac{n}{n+1}+(\mu_r-\muo)\frac{n}{n+1}\left(\dfrac{R}{R+t}\right)^{2n+1}]\mathcal{K}_1 +[\mu_r-\muo+\mu_r\frac{n}{n+1}+\muo]\mathcal{K}_2=0
\end{multline*}
Multiplying by $(n+1)$ we get,
\begin{multline*}
\left[\mu_r(n+1)+\muo{n}+(\mu_r-\muo){n}\left(\dfrac{R}{R+t}\right)^{2n+1}\right]\mathcal{K}_1 +\mu_r (2n+1)\mathcal{K}_2=0
\end{multline*}
Putting the value of \(\mathcal{K}_2\) from Eq.(\ref{k2}) we get,
\begin{multline*}
\left[\mu_r(n+1)+\muo{n}+(\mu_r-\muo){n}\left(\dfrac{R}{R+t}\right)^{2n+1}\right]\mathcal{K}_1 \\+\mu_r (2n+1) \left[-G_n-\frac{\mu_r(n+1)+\muo n}{(\mu_r-\muo)(n+1)}\mathcal{K}_1\right]=0
\end{multline*}

\begin{multline*}
\left[\frac{\mu_r^2 (n+1) (2n+1)+\muo\mu_r n(2n+1)}{(\mu_r-\muo)(n+1)}-\mu_r(n+1)-\muo{n}-(\mu_r-\muo){n}\left(\dfrac{R}{R+t}\right)^{2n+1}\right]\mathcal{K}_1\\=-\mu_r (2n+1)G_n
\end{multline*}

\begin{multline*}
[\mu_r^2 (n+1) (2n+1)+\muo\mu_r n(2n+1)-\mu_r^2(n+1)^2+\muo\mu_r(n+1)^2-\muo\mu_r n(n+1)
+\muo^2 n(n+1)\\-(\mu_r-\muo)^2 n(n+1)\left(\dfrac{R}{R+t}\right)^{2n+1}]\mathcal{K}_1=-\mu_r (\mu_r-\muo)(n+1)(2n+1)G_n
\end{multline*}


\begin{multline*}
[\mu_r^2(2n^2+2n+n+1-n^2-2n-1)+\muo\mu_r (2n^2+n-n^2-n)+\muo\mu_r(n+1)^2+\muo^2 n(n+1)\\-(\mu_r-\muo)^2 n(n+1)\left(\dfrac{R}{R+t}\right)^{2n+1}]\mathcal{K}_1=-\mu_r(\mu_r-\muo)(n+1) (2n+1)G_n
\end{multline*}

\begin{multline*}
[\mu_r^2 n(n+1)+\muo\mu_r n^2+\muo\mu_r(n+1)^2+\muo^2 n(n+1)-(\mu_r-\muo)^2 n(n+1)\left(\dfrac{R}{R+t}\right)^{2n+1}]\mathcal{K}_1\\=-\mu_r(\mu_r-\muo)(n+1) (2n+1)G_n
\end{multline*}

\begin{multline*}
[\mu_r n(\mu_r(n+1)+\muo n)+\muo(n+1)(\mu_r(n+1)+\muo n)-(\mu_r-\muo)^2 n(n+1)\left(\dfrac{R}{R+t}\right)^{2n+1}]\mathcal{K}_1\\=-\mu_r(\mu_r-\muo)(n+1) (2n+1)G_n
\end{multline*}

\begin{equation}\label{k1}
\mathcal{K}_1=-\frac{\mu_r(\mu_r-\muo)(n+1) (2n+1)G_n}{[\mu_r(n+1)+\muo n)][\mu_r n+\muo(n+1)]-(\mu_r-\muo)^2 n(n+1)\left(\dfrac{R}{R+t}\right)^{2n+1}}
\end{equation}
Now, from Eq. (\ref{R<r<R+t}), the net field  within the bulk of the shield (\textit{i.e.}, $R<r<R+t$)  is 
\begin{multline*}
\bm B_2= \left[\mathcal{K}_1 n(n+1)\frac{R^{2n+1}}{r^{n+2}}+(\mathcal{K}_2+G_n)n(n+1) r^{n-1}\right] P_n(u) \bm{\hat{r}}\\ + \left[\mathcal{K}_1 n\frac{R^{2n+1}}{r^{n+2}}-(\mathcal{K}_2+G_n)(n+1) r^{n-1}\right] P_n^1(u) \bm{\hat{\theta}}
\end{multline*}
\begin{multline}\label{B2}
% \begin{equation}\label{B2}
\bm B_2= n(n+1)r^{n-1}\left[\mathcal{K}_1 \left(\dfrac{R}{r}\right)^{2n+1}+(\mathcal{K}_2+G_n)\right] P_n(u) \bm{\hat{r}} \\+r^{n-1} \left[\mathcal{K}_1 n\left(\dfrac{R}{r}\right)^{2n+1}-(\mathcal{K}_2+G_n)(n+1) \right] P_n^1(u) \bm{\hat{\theta}}
% \end{equation}
\end{multline}
For \(\mu_r\gg\muo\), from Eq.(\ref{k1}) we get,
\begin{equation}\label{k12}
\begin{split}
\mathcal{K}_1^\infty & =-\frac{\mu_r^2(n+1) (2n+1)G_n}{\mu_r^2 n(n+1)-\mu_r^2 n(n+1)\left(\dfrac{R}{R+t}\right)^{2n+1}}\\
& =-\frac{(2n+1)G_n}{n-n\left(\dfrac{R}{R+t}\right)^{2n+1}}
\end{split}
\end{equation}


From Eq. (\ref{k2}) we get,
$$\(\mathcal{K}_2^\infty=-G_n-\frac{\mu_r(n+1)}{\mu_r(n+1}\mathcal{K}_1^\infty\)$$
$$\(\mathcal{K}_2^\infty=-G_n-\mathcal{K}_1^\infty\)$$
$$\(\mathcal{K}_2^\infty+G_n=-\mathcal{K}_1^\infty\)$$
Putting the value in Eq.(\ref{B2}) we get,
\begin{align*}
&\bm B_2= n(n+1)r^{n-1}[\mathcal{K}_1 \left(\dfrac{R}{r}\right)^{2n+1}-\mathcal{K}_1] P_n(u) \bm{\hat{r}} +r^{n-1} [\mathcal{K}_1 n\left(\dfrac{R}{r}\right)^{2n+1}+\mathcal{K}_1(n+1) ] P_n^1(u) \bm{\hat{\theta}}\\
&\bm B_2= n(n+1)r^{n-1}\mathcal{K}_1[\left(\dfrac{R}{r}\right)^{2n+1}-1] P_n(u) \bm{\hat{r}} +r^{n-1} \mathcal{K}_1[n\left(\dfrac{R}{r}\right)^{2n+1}+(n+1) ] P_n^1(u) \bm{\hat{\theta}}
\end{align*}

Now, using eq. (\ref{k12}) we get,
\begin{multline*}
\bm B_2= -n(n+1)r^{n-1}\left[-\frac{(2n+1)G_n}{n-n\left(\dfrac{R}{R+t}\right)^{2n+1}}][1-\left(\dfrac{R}{r}\right)^{2n+1}\right] P_n(u) \bm{\hat{r}} \\+r^{n-1} \left[-\frac{(2n+1)G_n}{n-n\left(\dfrac{R}{R+t}\right)^{2n+1}}\right]\left[(n+1)+n\left(\dfrac{R}{r}\right)^{2n+1}\right] P_n^1(u) \bm{\hat{\theta}}
\end{multline*}
\begin{multline*}
\bm B_2= (n+1)(2n+1)r^{n-1}G_n\left[\frac{1-\left(\dfrac{R}{r}\right)^{2n+1}}{1-\left(\dfrac{R}{R+t}\right)^{2n+1}}\right] P_n(u) \bm{\hat{r}} \\-(2n+1)r^{n-1}G_n\left [\frac{\frac{n+1}{n}+\left(\dfrac{R}{r}\right)^{2n+1}}{1-\left(\dfrac{R}{R+t}\right)^{2n+1}}\right] P_n^1(u) \bm{\hat{\theta}}
\end{multline*}
\begin{multline*}
\bm B_2= (n+1)(2n+1)r^{n-1}G_n\left[\frac{1-\left(\dfrac{R}{r}\right)^{2n+1}}{1-\left(\dfrac{R}{R+t}\right)^{2n+1}}\right] P_n(u) \bm{\hat{r}} \\-\frac{(2n+1)}{n}(n+1)r^{n-1}G_n \left[\frac{1+\frac{n}{n+1}\left(\dfrac{R}{r}\right)^{2n+1}}{1-\left(\dfrac{R}{R+t}\right)^{2n+1}}\right] P_n^1(u) \bm{\hat{\theta}}
\end{multline*}

Therefore,

$$\(\bm B_2= (n+1)(2n+1)r^{n-1}G_n\left[\left[\frac{1-\left(\dfrac{R}{r}\right)^{2n+1}}{1-\left(\dfrac{R}{R+t}\right)^{2n+1}}\right] P_n(u) \bm{\hat{r}} -\frac{1}{n}\left[\frac{1+\frac{n}{n+1}\left(\dfrac{R}{r}\right)^{2n+1}}{1-\left(\dfrac{R}{R+t}\right)^{2n+1}}\right] P_n^1(u) \bm{\hat{\theta}}\right]\)$$


\subsection{Uniform internal field generation}
From ~\cite{ smythe}, the field in the region r\textless a for a current loop at \(\theta=\alpha\) is,
\begin{equation}\label{bbr}
B_r=\frac{\muo I \sin \alpha}{2a}\sum_{n=1}^{\infty} \left(\frac{r}{a}\right)^{n-1}P_n^1(\cos\alpha) P_n(\cos\theta)
\end{equation}
\begin{equation}\label{bbt}
B_\theta=-\frac{\muo I \sin \alpha}{2a}\sum_{n=1}^{\infty} \frac{1}{n}\left(\frac{r}{a}\right)^{n-1}P_n^1(\cos\alpha) P_n^1(\cos\theta)
\end{equation}

For Helmholtz coils of radius \(r_c\) located at \(z=\pm z_o\) i.e. for two loops at \(\theta=\alpha, \pi-\alpha \), \(I_2=I_1=I\) and \(z_o=\frac{r_c}{2}\),
\begin{equation}\label{sinH}
\sin\alpha=\frac{r_c}{a}=\frac{r_c}{\sqrt{z_o^2+r_c^2}}=\frac{r_c}{\sqrt{\left(\frac{r_c}{2}\right)^2+r_c^2}}=\frac{r_c}{\sqrt{\frac{5r_c^2}{4}}}=\frac{2}{\sqrt{5}}
\end{equation}
\begin{equation}\label{sinHb}
\sin(\pi-\alpha)=\sin\pi\cos\alpha-\cos\pi\sin\alpha=\sin\alpha=\frac{2}{\sqrt{5}}
\end{equation}

\begin{equation}\label{cosH}
\cos\alpha=\frac{z_o}{a}=\frac{r_c}{2\sqrt{z_o^2+r_c^2}}=\frac{r_c}{2\sqrt{\left(\frac{r_c}{2}\right)^2+r_c^2}}=\frac{r_c}{2\sqrt{\frac{5r_c^2}{4}}}=\frac{1}{\sqrt{5}}
\end{equation}
\begin{equation}\label{cosHb}
\cos(\pi-\alpha)=\cos\pi\cos\alpha+\sin\pi\sin\alpha=-\cos\alpha=-\frac{1}{\sqrt{5}}
\end{equation}


So putting the values from eq. [\ref{sinH} and \ref{cosH}] in eq. [\ref{bbr}] we get,
$$B_r^{top}=\frac{\muo I 2}{2a\sqrt{5}}\sum_{n=1}^{\infty} \left(\frac{r}{a}\right)^{n-1}P_n^1\left(\frac{1}{\sqrt{5}}\right) P_n(\cos\theta)$$
So,
\begin{equation}\label{bbrH}
B_r^{top}=\frac{\muo I }{\sqrt{5}a}\sum_{n=1}^{\infty} \left(\frac{r}{a}\right)^{n-1}P_n^1\left(\frac{1}{\sqrt{5}}\right) P_n(\cos\theta)
\end{equation}
Again, putting the values from eq. [\ref{sinHb} and \ref{cosHb}] in eq. [\ref{bbr}] we get,
\begin{equation}\label{bbrHb}
B_r^{bottom}=\frac{\muo I }{\sqrt{5}a}\sum_{n=1}^{\infty} \left(\frac{r}{a}\right)^{n-1}P_n^1\left(-\frac{1}{\sqrt{5}}\right) P_n(\cos\theta)
\end{equation}
Therefore,
\begin{equation}\label{brtot}
B_r=B_r^{top}+B_r^{bottom}
\end{equation}
Using eq. [\ref{bbrH} and \ref{bbrHb}],
$$B_r=\frac{\muo I }{\sqrt{5}a}\sum_{n=1}^{\infty} \left(\frac{r}{a}\right)^{n-1}P_n^1\left(\frac{1}{\sqrt{5}}\right) P_n(\cos\theta)+\frac{\muo I }{\sqrt{5}a}\sum_{n=1}^{\infty} \left(\frac{r}{a}\right)^{n-1}P_n^1\left(-\frac{1}{\sqrt{5}}\right) P_n(\cos\theta)$$
Again,
\begin{equation}\label{leg}
P_n^1(-u)=(-1)^{n+1}P_n^1(u)
\end{equation}
From eq.[\ref{leg}] we get that only the odd n terms contribute to net field of eq.[\ref{brtot}]. Furthermore, since \(P_3^1\left(\pm\frac{1}{\sqrt{5}}\right)\) is uniquely zero – thereby eliminating the n = 3.


So we get,
\begin{equation}\label{brtotf}
B_r=\frac{2 \muo I }{\sqrt{5}a}\sum_{n=1,5,7,...}^{\infty} \left(\frac{r}{a}\right)^{n-1}P_n^1\left(\frac{1}{\sqrt{5}}\right) P_n(\cos\theta)
\end{equation}
Similarly, putting the values from eq. [\ref{sinH} and \ref{cosH}] in eq. [\ref{bbt}] we get,
$$B_\theta^{top}=-\frac{\muo I 2}{2a\sqrt{5}}\sum_{n=1}^{\infty} \frac{1}{n}\left(\frac{r}{a}\right)^{n-1}P_n^1\left(\frac{1}{\sqrt{5}}\right) P_n^1(\cos\theta)$$
So,
\begin{equation}\label{bbtH}
B_\theta^{top}=-\frac{\muo I }{\sqrt{5}a}\sum_{n=1}^{\infty}\frac{1}{n} \left(\frac{r}{a}\right)^{n-1}P_n^1\left(\frac{1}{\sqrt{5}}\right) P_n^1(\cos\theta)
\end{equation}

Again, putting the values from eq. [\ref{sinHb} and \ref{cosHb}] in eq. [\ref{bbt}] we get,
\begin{equation}\label{bbtHb}
B_\theta^{bottom}=-\frac{\muo I }{\sqrt{5}a}\sum_{n=1}^{\infty}\frac{1}{n} \left(\frac{r}{a}\right)^{n-1}P_n^1\left(\frac{1}{\sqrt{5}}\right) P_n^1(\cos\theta)
\end{equation}

Therefore,
\begin{equation}\label{bttot}
B_\theta=B_\theta^{top}+B_\theta^{bottom}
\end{equation}
Using eq. [\ref{bbtH} and \ref{bbtHb}],
$$B_\theta=-\frac{\muo I }{\sqrt{5}a}\sum_{n=1}^{\infty}\frac{1}{n} \left(\frac{r}{a}\right)^{n-1}P_n^1\left(\frac{1}{\sqrt{5}}\right) P_n^1(\cos\theta)-\frac{\muo I }{\sqrt{5}a}\sum_{n=1}^{\infty}\frac{1}{n} \left(\frac{r}{a}\right)^{n-1}P_n^1\left(\frac{1}{\sqrt{5}}\right) P_n^1(\cos\theta)$$

As discussed above we get,
\begin{equation}\label{brtotf}
B_\theta=-\frac{2 \muo I }{\sqrt{5}a}\sum_{n=1,5,7,...}^{\infty}
\frac{1}{n}\left(\frac{r}{a}\right)^{n-1}P_n^1\left(\frac{1}{\sqrt{5}}\right) P_n^1(\cos\theta)
\end{equation}



Now again for Anti-Helmholtz coils of radius \(r_c\) located at \(z=\pm z_o\) i.e. for two loops at \(\theta=\alpha, \pi-\alpha \), \(I_2=-I_1=I\) and \(z_o=\sqrt{3}\frac{r_c}{2}\),
\begin{equation}\label{sinAH}
\sin\alpha=\frac{r_c}{a}=\frac{r_c}{\sqrt{z_o^2+r_c^2}}=\frac{r_c}{\sqrt{\left(\frac{\sqrt{3}r_c}{2}\right)^2+r_c^2}}=\frac{r_c}{\sqrt{\frac{7r_c^2}{4}}}=\frac{2}{\sqrt{7}}
\end{equation}
\begin{equation}\label{sinAHb}
\sin(\pi-\alpha)=\sin\pi\cos\alpha-\cos\pi\sin\alpha=\sin\alpha=\frac{2}{\sqrt{7}}
\end{equation}
\begin{equation}\label{cosAH}
\cos\alpha=\frac{z_o}{a}=\frac{\sqrt{3}r_c}{2\sqrt{z_o^2+r_c^2}}=\frac{\sqrt{3}r_c}{2\sqrt{\left(\frac{\sqrt{3}r_c}{2}\right)^2+r_c^2}}=\frac{\sqrt{3}r_c}{2\sqrt{\frac{7r_c^2}{4}}}=\sqrt{\frac{3}{7}}
\end{equation}
\begin{equation}\label{cosAHb}
\cos(\pi-\alpha)=\cos\pi\cos\alpha+\sin\pi\sin\alpha=-\cos\alpha=-\sqrt{\frac{3}{7}}
\end{equation}

So puting the values from eq. [\ref{sinAH} and \ref{cosAH}] in eq. [\ref{bbr}] we get,
$$B_{r(anti)}^{top}=\frac{\muo (-I) 2}{2a\sqrt{7}}\sum_{n=1}^{\infty} \left(\frac{r}{a}\right)^{n-1}P_n^1\left(\sqrt{\frac{3}{7}}\right) P_n(\cos\theta)$$
So,
\begin{equation}\label{bbrAH}
B_{r(anti)}^{top}=-\frac{\muo I }{\sqrt{7}a}\sum_{n=1}^{\infty} \left(\frac{r}{a}\right)^{n-1}P_n^1\left(\sqrt{\frac{3}{7}}\right) P_n(\cos\theta)
\end{equation}
Again, putting the values from eq. [\ref{sinAHb} and \ref{cosAHb}] in eq. [\ref{bbr}] we get,
\begin{equation}\label{bbrAHb}
B_{r(anti)}^{bottom}=\frac{\muo I }{\sqrt{7}a}\sum_{n=1}^{\infty} \left(\frac{r}{a}\right)^{n-1}P_n^1\left(-\sqrt{\frac{3}{7}}\right) P_n(\cos\theta)
\end{equation}


Therefore,
$$B_{r(anti)}=B_{r(anti)}^{top}+B_{r(anti)}^{bottom}$$
Using eq. [\ref{bbrAH} and \ref{bbrAHb}],
$$B_{r(anti)}=-\frac{\muo I }{\sqrt{7}a}\sum_{n=1}^{\infty} \left(\frac{r}{a}\right)^{n-1}P_n^1\left(\sqrt{\frac{3}{7}}\right) P_n(\cos\theta)+\frac{\muo I }{\sqrt{7}a}\sum_{n=1}^{\infty} \left(\frac{r}{a}\right)^{n-1}P_n^1\left(-\sqrt{\frac{3}{7}}\right) P_n(\cos\theta)$$

From eq.[\ref{leg}] we get that only the even n terms contribute to net field of eq.[\ref{brtot}]. So, we get,
\begin{equation}\label{brtotAH}
B_{r(anti)}=-\frac{2 \muo I }{\sqrt{7}a}\sum_{n=2,4,6,...}^{\infty}
\left(\frac{r}{a}\right)^{n-1}P_n^1\left(\sqrt{\frac{3}{7}}\right) P_n(\cos\theta)
\end{equation}


Similarly,puting the values from eq. [\ref{sinAH} and \ref{cosAH}] in eq. [\ref{bbt}] we get,
$$B_{\theta(anti)}^{top}=-\frac{\muo (-I) 2}{2a\sqrt{7}}\sum_{n=1}^{\infty} \frac{1}{n}\left(\frac{r}{a}\right)^{n-1}P_n^1\left(\sqrt{\frac{3}{7}}\right) P_n^1(\cos\theta)$$
So,
\begin{equation}\label{bbtAH}
B_{\theta(anti)}^{top}=\frac{\muo I }{\sqrt{7}a}\sum_{n=1}^{\infty} \frac{1}{n}\left(\frac{r}{a}\right)^{n-1}P_n^1\left(\sqrt{\frac{3}{7}}\right) P_n^1(\cos\theta)
\end{equation}
Again, putting the values from eq. [\ref{sinAHb} and \ref{cosAHb}] in eq. [\ref{bbt}] we get,
\begin{equation}\label{bbtAHb}
B_{\theta(anti)}^{bottom}=-\frac{\muo I }{\sqrt{7}a}\sum_{n=1}^{\infty}\frac{1}{n} \left(\frac{r}{a}\right)^{n-1}P_n^1\left(-\sqrt{\frac{3}{7}}\right) P_n^1(\cos\theta)
\end{equation}


Therefore,
$$B_{\theta(anti)}=B_{\theta(anti)}^{top}+B_{\theta(anti)}^{bottom}$$
Using eq. [\ref{bbtAH} and \ref{bbtAHb}],
$$B_{\theta(anti)}=\frac{\muo I }{\sqrt{7}a}\sum_{n=1}^{\infty} \frac{1}{n}\left(\frac{r}{a}\right)^{n-1}P_n^1\left(\sqrt{\frac{3}{7}}\right) P_n^1(\cos\theta)-\frac{\muo I }{\sqrt{7}a}\sum_{n=1}^{\infty}\frac{1}{n} \left(\frac{r}{a}\right)^{n-1}P_n^1\left(-\sqrt{\frac{3}{7}}\right) P_n^1(\cos\theta)$$

As discussed above we get,
\begin{equation}\label{bttotAH}
B_{\theta(anti)}=\frac{2 \muo I }{\sqrt{7}a}\sum_{n=2,4,6,...}^{\infty}
\frac{1}{n}\left(\frac{r}{a}\right)^{n-1}P_n^1\left(\sqrt{\frac{3}{7}}\right) P_n^1(\cos\theta)
\end{equation}